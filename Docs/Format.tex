\documentclass[english]{article}
\usepackage[T1]{fontenc}
\usepackage[latin1]{inputenc}
\setlength\parskip{\medskipamount}
\setlength\parindent{0pt}
\usepackage{array}
\usepackage{rotating}
\usepackage{hyperref}
\usepackage{longtable}
\usepackage{times}

\makeatletter

%%%%%%%%%%%%%%%%%%%%%%%%%%%%%% LyX specific LaTeX commands.
\newcommand{\noun}[1]{\textsc{#1}}
%% Bold symbol macro for standard LaTeX users
\newcommand{\boldsymbol}[1]{\mbox{\boldmath $#1$}}


\usepackage{babel}
\makeatother

\setlength{\textwidth}{155mm}
\setlength{\oddsidemargin}{5mm}
\setlength{\evensidemargin}{5mm}
\pagestyle{myheadings}
\markboth{GRAFFER}{Version 5.00}

\begin{document}

\title{\includegraphics[width=0.80\textwidth]{logo} \\
A Flexible Electronic Graph Paper\\
Version 5.00\\
File Format Description}

\author{\textsf{\textbf{\Large James Tappin}}\\
\texttt{\textbf{\Large jtappin at gmail dot com}}}

\date{\textsf{\textbf{\large May 2022}}}

\maketitle

\tableofcontents{}
\section{Introduction.}

Graffer files are tagged datasets which contain all the information
needed to describe the plot, its layout on the page, how to print
it etc. A major objective in designing the format was that there should
be no need for any extra files to be transferred along with the graffer
file (thus all the data are incorporated into the file).

There are two distinct graffer file formats, binary and ASCII. It is
intended that binary files should be used as the normal format, and
ASCII files should be used to transfer between computers which have a
different representation of numbers. Historically binary files were not
portable as they use native number formats, not XDR format. However
since most machines use IEEE number representations and endianness can
be detected this is not normally an issue with current systems. They
are also smaller and quicker to read and write, but cannot (unless
you're really hot with your hex editor) be manually repaired.


\section{Generic Issues.}

All Graffer files start with a general header, that allows Graffer
to recognise that the file is a graffer file and whether it is binary
or ASCII. Fields within the file are identified by tags which precede
the field. Tags are from 1 to 3 uppercase letters and numbers. The
order of tags is not significant except that tags referring to a specific
object (dataset or free text annotation) are grouped together.

As is normal in IDL:

\begin{description}
\item [Int] is a short (2 byte) integer
\item [Long] is a long (4 byte) integer
\item [Byte] is a single byte. In Fortran this is usually a logical,
  but there are a few 1-byte ints.
\item [Float] is a short (4 byte) floating point number
\item [Double] is a long (8 byte) floating point number
\item [String] is a string of characters of variable length.
\item [Null] refers to a tag with no values associated with it.
\end{description}

This document could hardly be described as a formal definition of
the Graffer file format; but rather it tries to give enough information
to allow you to figure out what's going on. It should probably be
read alongside the source code of the reading and writing routines
(\texttt{gr\_get\_asc.pro}, \texttt{gr\_get\_bin.pro},
\texttt{gr\_asc\_save.pro}, 
\texttt{gr\_bin\_save.pro} and the routines which they call). It is
also useful to have at least a working knowledge of IDL's plotting
keywords and variables.


\subsection{Binary files.}
\begin{table}
  \centering
  \begin{tabular}{lrlll}
    \hline
    Field & Length & Type & ``Name'' & Content \\
    \hline
    1 & 7 & String & & ``GRAFFER''\\
    2 & 2 & Int & Vmaj & Major version number. \\
    3 & 2 & Int & Vmin & Minor version number. \\
    4 & 4 & Long & Ldir & Length of the directory name. \\
    5 & Ldir & String & Dirname & The directory name. \\
    6 & 4 & Long & Lfile & Length of the file name. \\
    7 & Lfile & String & Fname & The file name.\\
    8 & 4 & Long & Ldate & Length of the date string.\\
    9 & Ldate & String & Date & The date of writing.\\
    \hline
  \end{tabular}
  \caption{The GRAFFER header layout for a binary file.}
  \label{tab:header}
\end{table}
The header of a binary file is described in \autoref{tab:header}.
% \begin{quote}
%   \texttt{GRAFFER} \emph{<version number (2 Ints)> <length of directory
%     name (Long)> <directory name (String)> <length of file name (Long)>
%     <file name (String)> <length of time (Long)> <time of writing
%     (String)>}
% \end{quote}

Since we may reasonably assume that the major version number will not
exceed 256 (at the present rate of progress this will take over a
millenium), the magnitude of the version number is used to determine
whether the file needs to be byteswapped.

For binary files, tags are always padded with spaces to a length
of 3. 

\subsubsection{Version 5}
\label{sec:bin_v5}

The one major structural change justifying a major version increments
is the change in how the classic XY datasets are stored. Now the
internal data structure uses a structure with pointers to the X and Y
values and appropriate errors. This is reflected in the new \texttt{VX},
\texttt{VY}, \texttt{VXE}, and \texttt{VYE} tags in binary files.

\subsubsection{Version 4}
\label{sec:bin_v4}

For version~4 and above the format has been improved such that the file
is much more robust against bad tags, this will mean that in principle
files will be readable by older versions of Graffer than that which
wrote the file (this was not possible in earlier versions as any
unrecognized tag or change of variable size [e.g. int to long] would
confuse the binary reader completely). Each tag is
now followed by information which defines how much data and of what
type follows, the record structure is described in
\autoref{tab:graff_record}. Version~4 files are always written in
little-endian byte order (although the read routines will handle either
ordering).

\begin{table}
  \centering
  \begin{tabular}{llrp{0.6\textwidth}}
    \hline
    Name & Type & Number & Description \\
    \hline
    TAG & Str*3 & 1 & The field identification tag \\
    TYPE & Long & 1 & The type code as returned by IDL's \texttt{SIZE}
    function. TYPE=11, implies a LIST object, any other sort of object
    is an error.\\
    NDIMS & Long & 1 & How many dimensions, 0 for a scalar.\\
    DIMS & Long & NDIMS & The size of the dimensions (not present when
    NDIMS=0).\\
    LENS & Long & NVALS & The lengths of strings. NVALS = PRODUCT(DIMS)
    or 1 for a scalar. Not present if TYPE is not 7 (string).\\
    DATA & TYPE & NVALS & The actual values. For list types (TYPE=11)
    see below for the storage of list elements.\\
    \hline
    ETYPE & Long & 1 & The type of the list element. Note that lists of
    lists are not allowed.\\
    ENDIMS & Long & 1 & The number of dimensions  in the list
    element.\\
    EDIMS &  Long & ENDIMS & The size of the dimensions (not present if
    ENDIMS=0).\\
    ELENS & Long & ENVALS & The lengths of the strings if the element
    is of string type. ENVALS=PRODUCT(EDIMS) or 1.\\
    EDATA & TYPE & ENVALS & The actual values in the element.\\
    \hline
  \end{tabular}
  \caption{The subfields of a Graffer~V4+ record.}
  \label{tab:graff_record}
\end{table}

\subsubsection{Version 2 and 3}
\label{sec:bin_v23}

For versions~2 and~3, the format was much less robust.
The size of the data field that follows is determined by the
type and number of data required by the particular tag. For string quantities,
the string is preceded by its length (a long). This became an issue if
the type of a field was changed (e.g.\ from float to double or int to long).

Version 2 and 3 files cannot be read directly by the Fortran version,
but the gdl/idl procedure \texttt{graff\_convert} can be used to
convert to the current format.

\subsubsection{Version 1}
\label{sec:vers1}

There were no binary files in version 1.

\subsection{ASCII files.}

The ASCII file header is:
\begin{quote}
  \texttt{Graffer V} \emph{<major version>}\texttt{.}\emph{<minor
    version>}\texttt{:}\emph{<directory><filename>}\texttt{: @}
  \emph{<time of writing>}
\end{quote}
as a single line. (The case of the word Graffer is used to decide
whether to try to read the file as binary or ASCII).
For ASCII files, tags and data are delimited by colons. Apart from
the header each record of the file must begin with a tag (which is
not preceded by a colon); multiple fields may occur in one record,
in which case the tags are preceded by colons. For obvious reasons,
a string field has special rules: a string field is terminated by
end of line, so a string quantity must be the last item on the line.


The formats used for the different variable types are not fixed as
some quantities have ranges which are much more limited than the type
that they are stored as. The format needs to leave spaces between quantities
as free-form reads are used after the line has been split at the colons.

\subsection{Version 1}
\label{sec:vers1}

Version 1 files were a fixed format file with no tags, and relatively
limited capabilities. Version 4.x will still read V1 files. They are
very rare nowadays. There were no binary files in version 1.


\section{General Tags}

The general tags are those which introduce quantities which apply
to the whole file.

\begin{longtable}{|llcp{9cm}|}
  \hline Tag& Type& Quantity&
  Meaning\\
  \hline
  \endhead
  \hline
  \endfoot
  GT& String& 1&
  Plot title (IDL's \noun{title} key)\\
  GS& String& 1&
  Plot subtitle (\noun{subtitle} key)\\
  GC& Float& 1&
  Character size for annotations, relative to default.\\
  GA& Float & 1&
  Thickness of axes on plot\\
  GP& Float& 4& Position of plot in Normalized coordinates (Format
  identical to the
  IDL \texttt{!P.Position} variable).\\
  GR& Float& 2& Aspect ratio of plot and smaller margin as a fraction
  of the page
  size. (Only one of GP and GR has non-zero values)\\
  GI & Byte & 1 & Whether the plot has isotropic axes.\\
  GHA & Byte & 1 & Whether the screen plot should use the hard copy
  aspect ratio.\\
  GF & Int & 1 & Font family for annotations (used in IDL only).\\
  \hline \multicolumn{4}{|c|}{The next group of tags are responsible
    for
    defining the X-axis properties}\\
  \hline XR& Double& 2&
  The range of the X-axis\\
  XL & Byte & 1 & Whether the X-axis has logarithmic scaling.\\
  XSI& Int& 1&
  X-axis style setting (The IDL \texttt{XSTYLE} keyword)\\
  XSE& Int& 1& X-axis extra style items (1$\Rightarrow$Omit minor ticks
  (only in old files, this is now replaced by the \texttt{XMN} tag);
  2$\Rightarrow$Place an axis at the origin; 4$\Rightarrow$Suppress
  annotations on the
  axis; 8$\Rightarrow$Annotate an origin axis (not implemented in
  Fortran); 16$\Rightarrow$Write Y-axis labels parallel to the Axis
  (Fortran only); Other bits unused at present)\\
  XSG& Int& 1& X-axis grid lines setting (0$\Rightarrow$No grid lines;
  other values
  $\Rightarrow$ Grid line in IDL linestyle (value -1)\\
  XST& Int& 1& X-axis time labelling options (lsb: time labelling on,
  next 2 storage
  unit, next 2 display unit (0=s, 1=m, 2=h, 3=d)\\
  XSZ& Float& 1&
  X-axis value for stored zero in time label. (Display units).\\
  XMJ & Int & 1 & The number of major intervals to use in the
  X-axis. (Not implemented in Fortran)\\
  XMS & Double & 1 & The spacing of the major intervals on the
  X-axis. (Not implemented in IDL).\\
  XMN & Int & 1 & The number of minor intervals to use on the X-axis.\\
  XLL & Long & 3 & Threshold for changing tick format in logarithmic
  plots (only used by Fortran).\\
  XFM & String & 1 & The format specifier to use (Overridden by time
  labelling options) \\
  XNV & Int & 1 & The number of tick values used (ascii files only).\\
  XVL & Double & <XNV> & A list of explicit major tick locations. If
  present, this overrides the value of XMJ. (Not implemented in Fortran)\\
  XT& String& 1&
  X axis label.\\
  \hline \multicolumn{4}{|c|}{All the X tags have corresponding tags
    starting
    with Y for the Y axis, and R for the secondary Y axis}\\
  \hline
  YIR & Byte & 1 & Whether the secondary Y axis is to be displayed.\\
  \hline
  ZT& Int& 1& Colour table number for 2-D datasets displayed as
  colour plots (IDL colour table indices), as of Version 3.08 this is a
  default used to initialize the table for the dataset or if the
  display is 8-bit. Not to be confused with the ``ZT'' key in a datset,
  which is contour thicknesses.\\
  ZG& Float& 1& Gamma setting for the colour table (since version 3.08
  this is a
  default setting).\\
  DN& Int& 1& The total number of datasets in the plot (Must precede
  any dataset
  definitions).\\
  DC& Int& 1& The currently selected dataset (NB Zero-based, unlike the
  values on
  the GUI which are 1 based)\\
  TN& Int& 1& The total number of Text annotation strings in the
  plot. (Not including
  standard plot \& axis titles).\\
  REM& Strings& n&
  A comment attached to the dataset, not plotted anywhere.\\
  & Long+Strings& 1+n&
  For V3 and ASCII files.\\
  \hline
  \multicolumn{4}{|c|}{The following tags define how hardcopy will be
    generated.}\\ 
  \hline HC& Byte& 1&
  Whether to use colour PostScript. (Fortran always uses colour)\\
  HE& Byte& 1&
  Whether to generate Encapsulated PostScript or PDF. 0=Normal
  PostScript, 1=Encapsulated PS, 2=PDF with page positioning, 3=PDF
  suitable for inclusion in \LaTeX. (Obsolete)\\
  HO& Byte& 1&
  Whether to plot in landscape (0) or portrait (1) mode\\
  HY& Byte & 1 & Whether to use RGB (0) or CMYK (1) colour
  representation. (Not implemented in Fortran).\\
  HP& Byte& 1&
  Whether to use A4 (0) or US Letter (1) paper\\
  HT& Byte& 1&
  Whether to put a timestamp on the plot.\\
  HS& Float& 2&
  The size of the plot (in cm)\\
  HD& Float& 2& The offset of the bottom left of the plot from the
  bottom left of the
  paper (as you would normally view the plot)\\
  HAB& String& 1&
  The part of the spooling command before the filename\\
  HAA& String& 1&
  The part of the spooling command after the filename.\\
  HVB& String& 1&
  The part of the view command before the filename\\
  HVA& String& 1&
  The part of the view command after the filename.\\
  HPB & String & 1 &
  The part of the PDF view command before the filename.\\
  HPA & String & 1 &
  The part of the PDF view command after the filename.\\
  HPP & Byte & 3 &
  Whether to prompt before spooling or viewing the output file.\\
  HFN& String & 1 & The filename for the hardcopy output.\\
  HPS & String & 1 & The device for producing PostScript output. This
  and the other devices are only used in Fortran.\\
  HEP & String & 1 & The device for producing encapsulated PostScript output.\\
  HPD & String & 1 & The device for producing PDF output.\\
  HSV & String & 1 & The device for producing SVG output.\\

  HF& Int& 1& The font family (for hardware fonts) (IDL): 0=courier,
  1=helvetica, 2=helvetica narrow, 3=schoolbook, 4=palatino, 5=times,
  6=avantgarde book, 7=avantgarde demi, 8=bookman demi, 9=bookman
  light, 10=zapfchancery, 11=zapfdingbats, 12=symbol (Fortran): 1=Sans
  Serif, 2=Serif, 3=Monospaced, 4=Script, 5=Symbol. These are the fonts
  used for titles, axis annotations and keys in hard-copy output,
  manual annotations have their own font
  settings.\\
  HWS& Int& 1&
  The weight (0 normal, 1 bold) and slant (0 roman, 2 italic).\\
  \hline \multicolumn{4}{|c|}{The next group of tags control the
    display of a key or legend on the
    plot.}\\
  \hline KU& Byte& 1&
  Whether to display a key.\\
  KX& Double& 2&
  The x range in which to place the key\\
  KY& Double& 2&
  The y range in which to place the key\\
  KN& Int& 1& The coordinate system for the placement (0=data [always
  the primary Y-axis], 1=NDC\footnote{In practice, NDC coordinates in
    GRAFFER are relative to the plotting region, rather than the whole
    drawing surface}, 2=frame [Normalized
  coordinates relative to the axes]).\\
  KC& Int& 1&
  How many columns to use in the layout.\\
  KF& Byte& 1&
  Whether to put a box around the key\\
  KP& Byte& 1&
  Whether to plot 2 (0) or 1 (1) points on the example\\
  KS & Float & 1 & Character size for the key.\\
  KT& String& 1&
  Title for the key\\
  KL& Int& n&
  How many datasets in the key and which datasets to include.\\
  & Int& 1+n&
  For V3 and ASCII.\\
\end{longtable}



\section{Dataset Tags}

These tags define a specific dataset within the plot.

\begin{longtable}{|llcp{8cm}|}
  \hline Tag& Type& Quantity&
  Meaning\\
  \hline
  \endhead
  \hline
  \endfoot
  DS& Int& 1& Start a dataset, must come after the DN tag. Value is the
  dataset
  number.\\
  J& Int& 1&
  Joining option, 0=none, 1=line, 2=histogram\\
  P& Int& 1&
  Symbol (IDL \texttt{PSYM} value + some extras)\\
  S& Float& 1&
  Symbol size (Relative to default)\\
  L& Int& 1&
  Line style (IDL linestyle values)\\
  C& Int& 1&
  Colour (Table as in PGPLOT). -1 denotes omit this dataset, -2
  denotes a custom colour.\\
  CV & Byte & 3 & A colour triple (R,G,B), only present if the colour
  index is -2.\\
  W& Float& 1&
  Line thickness\\
  O& Byte& 1&
  Whether to sort the X-axis values before plotting\\
  K& Byte& 1&
  Whether to clip at the plot limits (0=clip, 1=noclip)\\
  E& Byte& 1&
  Whether to allow editing of the dataset with the mouse\\
  D& String& 1&
  Description of the dataset (only displayed if a key is plotted)\\
  N& Long& 1&
  How many points in the dataset (on the X-axis for a 2-D dataset)\\
  N2& Long& 1&
  How many points on the Y axis of a 2-D dataset.\\
  T& Int& 1& Dataset type (+ve values specify observational sets, -ve
  functions) 0: Plain X Y dataset, 1: X Y and Y-errors, 2: X, Y and X
  errors, 3: X, Y and +Y,-Y errors, 4: X, Y and +X,-X errors, 5: X, Y
  andX \& Y errors, 6: X, Y, and X, -Y, +Y errors, 7: X, Y and -X,+X,Y
  errors, 8: X, Y, and -X,+X, -Y,+Y errors. 9: 2-D data. -1: $y=F(X)$, -2:
  $x=F(Y)$, -3: $x=F(T),
  y=G(T)$; -4: $z=F(X,Y)$\\
  M& Int& 1&
  Coordinate system (0: rectangular, 1: Polar radians, 2: Polar
  degrees).\\
  Y & Int & 1 & Which Y-axis to use (0: main, 1: secondary).\\
  ZF& Int& 1&
  2-D dataset format (Contours or Gray/Colour)\\
  ZNL& Int& 1&
  Number of levels for contour plot. In version 4, this is only present
  for automatic levels.\\
  ZL& Double& <ZNL>&
  The contour levels to use\\
  ZLM & Int & 1 & Mapping mode for automatic contour levels. 0=linear,
  1=log, 2=square root. \\
  ZLL & Null & 0 & Indicates that the contour levels follow on multiple
  lines. Only used in V4 ASCII files when ZNL is more than 5.\\
  ZNC& Int& 1&
  Number of colours to use for coloured contouring (Not present in V4
  binary files).\\
  ZC& List & <ZNC>&
  The colour indices to use. List elements are either integers or
  3-element byte arrays. From V4.09, this is not used in ASCII files,
  but can still be read.\\
  ZCL & Null & 0 & Indicates that the colours follow on multiple
  lines. Only used in V4 ASCII files when ZNC is more than 20. No
  longer used as of V4.09 but can still be read.\\
  ZCX & Int & <ZNC> & In ASCII files in V4.09 and above, the value
  gives the number of elements in each colour value. The values then
  follow, one colour per line on the remainaing <ZNC> lines. \\
  ZNT& Int & 1&
  The number of line thicknesses to use in contouring (Not present in
  V4 binary files).\\
  ZT& Float& <ZNT>&
  The line thicknesses to use.\\
  ZTL & Null & 0 & Indicates that the thicknesses follow on multiple
  lines. Only used in V4 ASCII files when ZNT is more than 20.\\
  ZNS& Int& 1&
  The number of linestyles to use in contouring (Not present in V4).\\
  ZS& Int& <ZNS>&
  The linestyles to use.\\
  ZSL & Null & 0 & Indicates that the linstyles follow on multiple
  lines. Only used in V4 ASCII files when ZNS is more than 20.\\
  ZCF& Int& 1&
  Whether to fill contours (1) or add downhill ticks (2).\\
  ZLI& Int& 1&
  Labelling of contours (Label every $\mathrm{<ZLI>^{th}}$ contour\\
  ZCS & Float & 1 & The character size (relative to default) used for
  contour labels.\\
  ZR& Double& 2&
  The range of Z values to display in gray/colour display\\
  ZP& Float& 1&
  The pixel size to use (in mm) for hardcopy.\\
  ZIL & Byte & 1 & Whether the colour scaling is
  logarithmic. 0=linear, 1=log, 2=square root.\\
  ZIN & Byte & 1 & Whether the colour map is inverted.\\
  ZSM & Byte & 1 & Whether to use a smooth (\texttt{plshades}) or
  ``blocky'' (\texttt{plimage}) image display. (only in Fortran)\\
  ZSN & Long & 1 & How many levels to use for smooth image display.\\
  ZM & Double & 1 & A value to use for regions of an image display that
  do not map from the data. Also used for non-finite values in smoothed
  display in Fortran.\\
  ZCT & Int & 1 & The colour table for this dataset (set to 1+colour
  table number, 0 is use default).\\
  ZCG & Float & 1 & The gamma value for the colour mapping.\\
  R& Float& 2 or 4& The range over which to plot a function dataset. (4
  values for 2-D
  function datasets).\\
  F& String& 1&
  Function (for function datasets other than parametric (-3))\\
  FX& String& 1&
  X-function for type -3 function datasets\\
  FY& String& 1&
  Y-function for type -3.\\
  VS& Double& <N>$\times$Cols& The actual data. Note that this tag and
  the other start data tags are not quite like other valued tags in that
  in an ASCII dataset the values must start on a new line and be in <N>
  rows with the number of columns
  determined by the type setting. Now only used in ASCII files.\\
  VE& Null& 0&
  End of data. For binary files this is just a verification. Not
  present in V4\\
  VX & Double & <N> & The X-values for a 1-D dataset.\\
  VY & Double & <N> & The Y-values for a 1-D dataset.\\
  VXE & Double & 1 or 2 $\times$ <N> & The X-errors for a 1-D dataset.\\
  VY & Double & 1 or 2 $\times$ <N> & The Y-errors for a 1-D dataset.\\
  ZX2 & Byte & 1 & Flag for 2-d X array (must precede the ZXS tag, not
  present in V4+ binaries).\\
  ZY2 & Byte & 1 & Flag for 2-d Y array (must precede the ZYS tag, not
  present in V4+ binaries).\\
  ZXS& Double& <N>&
  X values for a 2-D dataset (will be <N>$\times$<N2> if ZX2 flag is set) \\
  ZXE& Null& 0&
  End of X-values. Not present in V4+ binaries.\\
  ZYS& Double& <N2>&
  Y values for a 2-D dataset (will be <N>$\times$<N2> if ZY2 flag is set)\\
  ZYE& Null& 0&
  End of Y values. Not present in V4.\\
  ZZS& Double& <N>$\times$<N2>&
  Z values in 2-D dataset\\
  ZZE& Null& 0&
  End of Z values. Not present in V4.\\
  DE& Null& 0&
  End of Dataset.\\
\end{longtable}


\section{Annotations.}

An annotation is a text string which can be freely placed on the plot,
it is not associated with a specific dataset, nor is it a standard
title or part of a key.

\begin{longtable}{|llcp{8cm}|}
\hline 
Tag&
Type&
Quantity&
Meaning\\
\hline
\endhead
\hline
\endfoot
TS& Int& 1&
Start text string, value is string index.\\
TTS& Null& 0&
Start text template (not all the following keys are used in a template).\\
\hline 
C& Int& 1&
Colour of text string. -2 denotes a custom colour.\\
CV & Byte & 3 &
Colour triple for custom colours.\\
S& Float& 1&
Size of characters (relative to default)\\
O& Float& 1&
Orientation (degrees anticlockwise for L->R).\\
A& Float& 1& Alignment relative to anchor, 0 - starts at anchor,1 -
ends at anchor,
0.5 centred on anchor.\\
FF & Int & 1 & The font family (corresponds to the values of IDL's
\texttt{!p.font} system variable.\\
F& Int& 1&
IDL font number. In V3, -1 is hardware font (using the same font as
plot annotations).\\
W& Float& 1&
Line thickness to use for drawing.\\
X& Double& 1&
X position of anchor. (Not in template)\\
Y& Double& 1&
Y position of anchor. (Not in template)\\
N& Byte& 1& Coordinate system (0 = data coords, 1 = NDC, 2 = Frame coordinates
[Normalized coordinates in terms of the axis box]).\\
AX & Int & 1 & Which Y-axis to use for Data coordinates. (Not in template)\\
T& String& 1&
The actual text string.\\
TID & String & 1 & An identifying ID to help programmatic access.\\
\hline TE& Null& 0&
End the text description\\
TTE& Null& 0&
End the template.\\
\end{longtable}

\section{Fortran Colour Table}
\label{sec:colour_table}

The colour table file lists the colours for each colour table. 

The first byte indicate how many tables the file contains.

For each table there is then:
\begin{itemize}
\item A 32 byte table name, as a character string.
\item Red intensities, as 256 1-byte values.
\item Green intensities, 256 1-byte values.
\item Blue intensities, 256 1-byte values.
\end{itemize}

Note that this is not the same as the IDL colour table format where the
names are at the end.
\end{document}
